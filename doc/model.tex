% You may title this section "Methods" or "Models". 
% "Models" is not a valid title for PLoS ONE authors. However, PLoS ONE
% authors may use "Analysis" 
\section*{Analysis}
 
\subsection*{Gene frequencies associated with Duffy negativity}
The model targets two spatially-varying allele frequencies, the frequency of the Fyb mutation and the frequency of the silencing mutation in the promoter region within Fyb individuals. These are denoted $p_{ab}(x)$ and $p_0(x)$ respectively, where $x$ is a location. A third allele frequency, the frequency of the silencing mutation within Fya individuals, is modeled as a small constant denoted $p_1$. The model for these allele frequencies is as follows:
\begin{eqnarray*}
    m_{ab} \sim \textup{Normal}(0,10000) \\
    m_0 \sim \textup{Normal}(0,10000)\\
    \beta_{\textup{africa}} \sim \textup{Normal}(0,10000)\\
    p_1\sim\textup{Uniform}(0,.4)\\
    \\
    M_{ab}(x) = \beta_{\textup{africa}} 1_{x\textup{\ in\ africa}} + m_{ab}\\
    M_0(x) = m_0\\
    \\
    \phi_{ab}\sim\textup{Exponential}(.1)\\
    \phi_{0}\sim\textup{Exponential}(.1)\\
    \theta_{ab}\sim\textup{Exponential}(.1)\\
    \theta_{0}\sim\textup{Exponential}(.1)\\
    \nu_{ab}\sim\textup{Uniform}(0,3)\\
    \nu_{0}\sim\textup{Uniform}(0,3)\\
    V_{ab}\sim\textup{Exponential}(.1)\\
    V_0\sim\textup{Exponential}(.1)\\
    \\
    C_{ab}(x,y) = \phi_{ab}\textup{Mat\`ern}(d(x,y)/\theta_{ab};\nu_{ab})+V_{ab}1_{x=y}\\
    C_{0}(x,y) = \phi_{0}\textup{Mat\`ern}(d(x,y)/\theta_{0};\nu_{0})+V_{0}1_{x=y}\\
    \\
    f_{ab}\sim\textup{GP}(M_{ab},C_{ab})\\
    f_{0}\sim\textup{GP}(M_{0},C_{0})\\
    p_{ab}(x)=\textup{logit}^{-1}(f_{ab}(x)) \\
    p_{0}(x)=\textup{logit}^{-1}(f_{0}(x))     
\end{eqnarray*}
The mean function for $f_0$ simply returns a constant, $m_0$. The mean of $f_{ab}$ takes presence in Africa as a covariate with coefficient $\beta_{\textup{africa}}$, and also a constant term $m_{ab}$. 

Both fields use the Mat\`ern covariance function [cite Banerjee]. The range parameters of $f_0$ and $f_{ab}$ are $\theta_0$ and $\theta_{ab}$, respectively; the corresponding amplitude parameters are $\phi_0$ and $\phi_{ab}$; the degree-of-differentiability parameters are denoted $\nu_{0}$ and $\nu_{ab}$; and the nugget variances are $V_0$ and $V_{ab}$. The distance function $d$ gives the great-circle distance between its arguments.

The Gaussian random fields are converted to probabilities using the standard inverse logit link function.

\subsubsection*{Likelihoods} % (fold)
\label{sec:likelihood}

The a/b switch mutation happens with probability $P(b=1)=p_{ab}(x)$, which should be much higher for $x$ in Africa. Given that $b=1$, the silencing mutation in the promoter region happens with probability $p_0(x)$. Given that $b=0$, the silencing mutation happens with probability $p_1$, which is assumed to be a small, constant value. Hardy-Weinberg assumptions apply to the genotype frequencies.
\begin{description}
    % \item[lon,lat]: Standard 
    % \item[n]: Sample size
    % \item[africa]: Whether the point was taken in Africa
    % \item[data]: The type of data
    \item[gen*]: Genotype data. The haplotype frequencies are:
    \begin{description}
        \item[gena] $(1-p_{ab}(x))(1-p_1)$
        \item[genb] $p_{ab}(x)(1-p_0(x))$ 
        \item[gen0] $p_{ab}(x)p_0(x)$
        \item[gen1] $(1-p_{ab}(x))p_1$
    \end{description}
    The genotype frequencies (genaa, genab, etc.) can be obtained using the standard Hardy-Weinberg formulas. For example, the frequency of $genab$ is twice the product of the frequencies of $gena$ and $genb$, which is $2(1-p_{ab}(x))(1-p_1)p_{ab}(x)(1-p_0(x))$
    \item[phe*]: Phenotype data.
    \begin{description}
        \item[pheab] This can only happen if the genotype is genab.
        \item[phea] This can happen if the genotype is gena0, gena1 or genaa.
        \item[pheb] This can happen if the genotype is genb0, genb1 or genbb.
        \item[phe0] This can only happen if the phenotype is gen00, gen01 or gen11.
    \end{description}
    \item[aphe*]: Phenotype data, Fya+/- only.
    \begin{description}
        \item[aphea] This can happen if the genotype is genaa, genab, gena1 or gena0.
        \item[aphe0] The complement of aphea.
    \end{description}
    \item[bphe*] Phenotype data, Fyb+/- only.
    \begin{description}
        \item[bpheb] This can happen if the genotype is genbb, genab, genb0 or genb1.
        \item[bphe0] The complement of bpheb. 
    \end{description}
    \item[prom]: Molecular study looked only at promoter region.
    \begin{description}
        \item[prom0]: This can only happen if the genotype is gen00, gen01 or gen11.
        \item[promab]: The complement of pos0.
    \end{description}
\end{description}

The sampling distributions are assumed to be multinomial conditional on the appropriate individual phenotype or genotype probabilities described above. This likelihood completes the Bayesian probability model.

\subsection*{\emph{P. vivax} endemicity} 
Assuming that Duffy negative individuals are refractory to infection by \emph{P. vivax}, the proportion of the population at risk of \emph{P. vivax} infection at location $x$ is $1-p_n(x)$, where the probability $p_n$ that any given individual is phenotypically Duffy negative is 
\begin{equation}
    \label{eq:dn-freq} 
    p_n(x)=\left(p_{ab}(x)p_0(x)\right)^2+2p_{ab}(x)p_0(x)(1-p_{ab}(x))p_1+\left((1-p_{ab}(x))p_1\right)^2.
\end{equation}

The \emph{P. vivax} endemicity within the population at risk can be modeled as a space-time random field as in \cite{hay2008}:
\begin{eqnarray*}
    \beta_{i} \stackrel{\textup{\tiny iid}}{\sim} \textup{Normal}(0,10000)\\
    M_v(x) = \sum_i\beta_ic_i(x)\\
    \\
    \phi_v\sim\textup{Exponential}(.1)\\
    \theta_v\sim\textup{Exponential}(.1)\\
    \nu_v\sim\textup{Uniform}(0,3)\\
    \\
    \rho_t\sim \textup{Uniform}(0,1) \\
    \theta_t\sim \textup{Exponential}(.01) \\
    \upsilon_t\sim \textup{Uniform}(0,1)\\
    V_v\sim\textup{Exponential}(.1)\\   
    \\ C_v(x,t;y,s)=\tau^2\gamma(0)\frac{d(x,y)/\theta_v^{\gamma(|t-s|)}K_{\gamma(|t-s|)}(d(x,y)/\theta_v)}{2^{\gamma(|t-s|)-1}\Gamma(\gamma(|t-s|)+1)},\\ 
    % FIXME: Get \nu_v in here.
    \gamma(|t-s|) = \frac{1}{2\rho+2(1-\rho)\left[(1-\upsilon_v)e^{-|t-s|/\theta_t}+\upsilon_v\cos(2\pi |t-s|)\right]},\\
    \\
    f_v\sim\textup{GP}(M_v, C_v)\\
    p_v(x)=\textup{logit}^{-1}(f_{v}(x))
\end{eqnarray*} 
where $K_\gamma$ is the modified Bessel function of the second kind of order $\gamma$ and $\Gamma$ is the gamma function \cite{whatever you cited in PLOS paper.}  A space-time process is more appropriate than a space-only process due to the seasonal and rapid secular changes seen in malaria transmission dynamics \cite{blah}. 

\subsection*{Environmental inputs}
One of the functions $c_i(x)$ was the constant function $c_i(x)=1$. Globcover channels $11$, $14$, $20$, $30$, $40$, $60$, $110$, $120$, $130$, $140$, $150$, $160$, $170$, $180$ and $200$ were also used (see \textbf{CITE} for an explanation of these channels). The synoptic mean and annual, biannual and triannual synoptic Fourier amplitudes of the MODIS daytime land temperature product \textbf{CITE} were also used.

Nearest-neighbor interpolation was used to evaluate the environmental input surfaces away from pixel centroids; that is, their values were assumed constant over square pixels whose edge lengths were $1/120$ decimal degrees at the equator for MODIS and $1/360$ decimal degrees at the equator for Globcover. In some cases, \emph{P. vivax} survey locations fell within pixels where no information was available for one of the environmental inputs. In these cases the modal value in a three pixel by three pixel box centered on the nearest pixel was used.

\subsubsection*{Likelihoods}
Combining the models for Duffy gene frequencies and \emph{P. vivax} endemicity within the population fraction at risk of infection, the probability that an individual at location $x$ is infected with \emph{P. vivax} is
\begin{equation}
    \label{eq:pv-like}
    p_v(x)(1-p_n(x))
\end{equation}

\subsection*{Implementation} 
The model was implemented in the Python programming language \cite{python}. It was fitted with the Markov chain Monte Carlo algorithm \cite{gilks} using the open-source Bayesian analysis package PyMC \cite{pymc}. Dynamic traces \cite{gilks} are available upon request.